\chapter{Primary Write Client Main Loop}
\label{app:dada_pwc_main}

In this chapter, the Primary Write Client Main Loop API is specified
in detail.  The Primary Write Client (PWC) Main Loop implements the
PWC Data Flow Control described in Chapter~\ref{sec:pwc_main}.  The
PWC Main Loop is implemented as a function, {\tt dada\_pwc\_main},
which receives a pointer to a struct, {\tt dada\_pwc\_main\_t}, as its
only argument.  The member variables of this struct must be properly
set up before the Main Loop is entered, as described in the following
section.

\section{Initialization}

The Primary Write Client Main Loop structure is created as follows:
\begin{verbatim}
#include "dada_pwc_main.h"
dada_pwc_main_t* pwcm = dada_pwc_main_create ();
\end{verbatim}
After creation, the following member variables must be initialized:
\begin{enumerate}

\item {\bf {\tt pwc}} pointer to a {\tt dada\_pwc\_t} struct; 
   this struct is used to communicate between the thread that parses
   external commands from the Command Interface and the thread that
   runs the PWC Main Loop.  The {\tt pwc} member variable can simply
   be set as in the following example:
\begin{verbatim}
/* create the command communication structure */
pwcm->pwc = dada_pwc_create ();

/* start the control connection server */
if (dada_pwc_serve (pwcm->pwc) < 0)
  /* report an error message */
\end{verbatim}

\item {\bf {\tt log}} pointer to the {\tt multilog\_t} struct
	that will be used for status and error reporting

\item {\bf {\tt data\_block}} pointer to the {\tt ipcio\_t} struct 
	that is connected to the Data Block

\item {\bf {\tt header\_block}} pointer to the {\tt ipcio\_t} struct
	that is connected to the Header Block

\item {\bf {\tt start\_function}} pointer to the function that starts the 
  data transfer:
\begin{verbatim}
time_t start_function (dada_pwc_main_t*, time_t utc);
\end{verbatim}
  This function receives the UTC start time, {\tt utc} at which the
  data transfer should begin.  If {\tt utc} equals zero, the function
  should start the data transfer at the soonest available opportunity.
  This function should take care of everything required to start the
  data transfer, and should return the UTC of the first time sample to
  be transfered or zero on error.

\item {\bf {\tt buffer\_function}} pointer to the function that returns
  the next buffer to be written to the Data Block:
\begin{verbatim}
void* buffer_function (dada_pwc_main_t*, uint64_t* size);
\end{verbatim}
  This function should return the base address of the next buffer or
  the {\tt NULL} pointer on error.  The size of the buffer (in bytes)
  should be returned in the {\tt size} argument.

\item {\bf {\tt stop\_function}} pointer to the function that stops
  the data transfer:
\begin{verbatim}
int stop_function (dada_pwc_main_t*);
\end{verbatim}
  This function should perform any tasks required to stop the data
  transfer before returning to the idle state.  This function should return
  a value less than zero in the case of error.

\item {\bf {\tt context}} [optional] pointer to any additional information.
  Should the implementation of any of the above three functions
  require access to other information, a pointer to this information
  can be stored in the {\tt context} member variable and retrieved by
  casting this member inside the function. e.g.
\begin{verbatim}
  struct puma2_t {
    EdtDev* edt_p;
    pic_t pic;
    unsigned buf_size;
  };

  void* puma2_buffer_function (dada_pwc_main_t* pwcm, uint64_t* size)
  {
    struct puma2_t* xfer = (struct puma2_t*) pwcm->context;
    *size = xfer->buf_size;
    return edt_wait_for_buffers (xfer->edt_p, 1);
  }

  [...]

  struct puma2_t xfer_data;

  pwcm->context = &xfer_data;
  pwcm->buffer_function = puma2_buffer_function;

  [etc...]
\end{verbatim}

\end{enumerate}
