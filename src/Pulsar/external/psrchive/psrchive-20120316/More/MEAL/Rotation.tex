\documentclass[12pt]{article}

% bold math italic font
\newcommand{\mbf}[1]{\mbox{\boldmath $#1$}}
\newcommand{\mbfs}[1]{\mbox{\scriptsize\boldmath $#1$}}

% equations
\newcommand{\Eqn}[1]{Equation~(\ref{eqn:#1})}
\newcommand{\Eqns}[1]{Equations~(\ref{eqn:#1})}
\newcommand{\eqn}[1]{equation~(\ref{eqn:#1})}
\newcommand{\eqns}[1]{equations~(\ref{eqn:#1})}

% symbol used for sqrt(-1)
\newcommand{\Ci}{\ensuremath{i}}

\newcommand{\var}{{\rm var}}
\newcommand{\trace}{{\rm tr}}
\newcommand{\sinc}{\,{\rm sinc}\,}

\newcommand{\Rotation}{{\bf R}}
\newcommand{\Boost}{{\bf B}}

\newcommand{\vRotation}[1][n]{\ensuremath{\Rotation_{\mbfs{\hat #1}}}}
\newcommand{\vBoost}[1][m]{\ensuremath{\Boost_{\mbfs{\hat #1}}}}

\newcommand{\rotat}{\ensuremath{\vRotation(\phi)}}
\newcommand{\boost}{\ensuremath{\vBoost(\beta)}}

\newcommand{\pauli}[1]{\mbf{\sigma_{#1}}}
\newcommand{\selection}[1]{\mbf{\delta_{#1}}}

\begin{document}

\noindent
The {\tt MEAL::Rotation} class represents the unitary matrices,
\begin{equation}
\label{eqn:Rotation}
\rotat = \exp (\Ci\mbf{\sigma\cdot\hat{n}}\phi)
       = \pauli{0}\cos\phi + \Ci\mbf{\sigma\cdot\hat{n}}\sin\phi\,
\end{equation}
which rotate the Stokes polarization vector about the axis
\mbf{\hat{n}} by an angle $2\phi$.  The rotation is parameterized by a
3-vector,
\begin{equation}
\mbf{r}=\mbf{\hat{n}}\phi = \sum_{j=1}^3 r_j \mbf{\hat{s}_j}
\end{equation}
such that $\phi=|\mbf{r}|$,
\begin{equation}
\rotat = \pauli{0}\cos\phi + \Ci\mbf{\sigma\cdot r}\sinc\phi,
\end{equation}

\begin{equation}
{\partial\phi\over\partial r_j} = {r_j\over\phi},
\end{equation}
and
\begin{equation}
{\partial\rotat\over\partial r_j} = -\pauli{0}r_j\sinc\phi 
+ \Ci \mbf{\sigma\cdot}\left[ \mbf{\hat{s}_j}\sinc\phi + \mbf{r} {r_j\over\phi^2} (\cos\phi -\sinc\phi)\right].
\end{equation}

\end{document}
