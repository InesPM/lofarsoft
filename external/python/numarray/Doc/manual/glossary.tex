\chapter{Glossary}
\label{cha:glossary}

\begin{quote} 
   This chapter provides a glossary of terms.\footnote{Please let us know of
      any additions to this list which you feel would be helpful.}
\end{quote}

\begin{description}
\item[array] An array refers to the Python object type defined by the NumPy
   extensions to store and manipulate numbers efficiently.
\item[byteswapped]
\item[discontiguous]  
\item[misaligned] 
\item[misbehaved array] A \class{\numarray} which is byteswapped, misaligned,
   or discontiguous.
\item[rank] The rank of an array is the number of dimensions it has, or the
   number of integers in its shape tuple.
\item[shape] Array objects have an attribute called shape which is necessarily
   a tuple. An array with an empty tuple shape is treated like a scalar (it
   holds one element).
\item[ufunc] A callable object which performs operations on all of the elements
   of its arguments, which can be lists, tuples, or arrays. Many ufuncs are
   defined in the umath module.
\item[universal function] See ufunc.
\end{description}
 


%% Local Variables:
%% mode: LaTeX
%% mode: auto-fill
%% fill-column: 79
%% indent-tabs-mode: nil
%% ispell-dictionary: "american"
%% reftex-fref-is-default: nil
%% TeX-auto-save: t
%% TeX-command-default: "pdfeLaTeX"
%% TeX-master: "numarray"
%% TeX-parse-self: t
%% End:
